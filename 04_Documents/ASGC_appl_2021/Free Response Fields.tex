\documentclass{article}
\usepackage[utf8]{inputenc}
\usepackage[letterpaper, portrait, margin=1in]{geometry}
\usepackage{times} % assumes new font selection scheme installed
\widowpenalty10000
\clubpenalty10000
\usepackage[font=footnotesize]{caption}

\begin{document}

\section{\emph{NASA Extramural Experience}} 

\small{\emph{(An extramural NASA experience is required sometime during the fellowship year. The details are to be worked out with the faculty advisor and NASA lab facility where the student plans to participate in a research activity. It is advisable that a NASA contact be found prior to submission of the fellowship proposal. In 400 words or less, please describe the extramural NASA experience that you plan to participate in. Names and locations of NASA field centers are: Ames Research Center - CA, Armstrong Flight Research Center - CA, Glenn Research Center - OH, Goddard Space Flight Center - MD, Jet Propulsion Laboratory - CA, Johnson Space Center - TX, Kennedy Space Center -- FL, Langley Research Center - VA, Marshall Space Flight Center - AL, and Stennis Space Center - MS.)}}
\\
\\
\large{I plan to conduct an extramural experience during the Summer of 2022 (2.5 months from June 1st) at the NASA Marshall Space Flight Center in Alabama. 
Researchers including Hill et. al. at NASA MSFC have recently developed a novel flexible hub for astronaut bio-sensing that will enable EMG electrodes to be used comfortably in EVA settings. This research is well-aligned with the project I have proposed, which involves the development of a passive instrumented glove for neuromuscular sensing during extravehicular activities. This extramural experience will directly contribute to my project in that, through the experience, I will be able to obtain and incorporate Hill et.al.'s sensor hub into the prototype of the proposed instrumented glove and study the manufacturing used by Hill et. al. to inform the design of the second iteration of the glove. I would like to further work with Hill et. al. to develop a second sensor hub based on their design which would allow the instrumented glove device I proposed to be fabricated all at once through 3D printing techniques rather than requiring separate assembly. The developments would substantially improve its quality and viability for use in EVA environments and would inform future approaches to astronaut bio-sensing.
Further, working with researchers at NASA MSFC will empower me with unique, area-specific knowledge and background that I could not obtain anywhere else and inform my intended career in developing human-centered robotics for space exploration. Collaborating with MSFC Researchers will shape and direct my knowledge of the design methodology for robust, flexible microelectronics and sensor design for extreme environments. By observing Hill et. al.’s methodology, I will be able to understand the principles of design associated with printing and fabricating flexible microelectronics which will enhance the work product I am able to produce in the future. The work of Hill et. al., Rogers et. al., and Madden et. al. among others suggests that there remains a great need for effective, intuitive, and robust biosensing to support long term human exploration and habitation in extreme environments. By obtaining this knowledge, I will be better positioned to address this need.
Thus, an extramural experience at NASA MSFC this summer would support and enhance the research I have proposed and would enable me to make relevant and robust contributions to long-term human habitation in orbital, Lunar, and Martian environments.}

\clearpage

\section{\emph{Science Education Outreach}} 

\small{\emph{Science Education Outreach	Fellows are expected to be involved in ASGC outreach activities. Proposed activities may take advantage of local opportunities and should involve the student's transmitting their knowledge and enthusiasm of science, math or technology to children or general audiences. These activities will differ from campus to campus and a specific assignment will be made after consultation with their ASGC campus director. In 400 words or less, please describe your proposed science education outreach activities.}}
\\
\\
\large
If selected for funding, I will be best able to dedicate time towards developing robotics education tools for K-12 aged students in collaboration with the Auburn Southeastern Center of Robotics Education (SCORE). I will undertake two endeavours in this regard.

SCORE has assembled a 3D printing lab to support the incorporation of engineering, hardware design and computer aided modeling in the classroom. Students will thereby enter higher education and join the workforce with a foundation in 3D printing techniques, which are becoming ubiquitous in every many industries. A similar emerging technique is simple at-home electronics and circuit design and prototyping. Several products and services already allow students to use quick-drying electrostatic ink pens to 'draw' active circuits on paper that can interface microelectronic implements like coin cell batteries, light emitting diodes, and microcontrollers. During year one of funding, I will work with SCORE-affiliated academic programs and coordinate with manufacturers and distributors to produce a 3D printed microelectronics demonstration. This demonstration will include the design, fabrication and use of a simple electromyographic (EMG) sensing module consisting of a single surface electrode and its associated positive and negative terminals. The sensor will be printed on a flexible substrate using methods developed at NASA's Marshal Space Flight Center and will help students better understand the capabilities of 3D printing in the realm of electronics as well as in rapid hardware prototyping. This outreach is very closely related to the research I have proposed as several similar flexible EMG modules will be included in the product of the research. 

During years two and three, if funded, I will work with SCORE-affiliated schools to develop and provide an assistive tool for K-12 aged students with disabilities to participate in the fine arts. I will adjust the software of the neuromuscular sensing glove described in my research proposal to enable students with disabilities to interface with digital instruments for audiovisual therapy. This builds on the research of Professor Andy Hunt at the University of New York, who validated the benefits of myographic sensing devices in music therapy towards the cognitive development and physical health of children, and Dr. Edgar Hunt at Louisiana State University who has worked to develop numerous haptic approaches to digital music production.

Thus, the funding provided by ASGC will help me support free robotics-oriented K-12 curriculum content and develop assistive systems to enhance the cognitive and physical health of K-12 aged students.



\clearpage

\section{\emph{Schedule of Target Dates}} 

\small{\emph{Please realistically identify the starting and completion dates for the proposed research program or plan of study, including the expected date for completion of the formal degree program. Any time expected to be spent at a NASA facility, including the required extramural experience, should be taken into consideration in establishing the target dates.}}
\\
\\
\large
08/2021 - 12/2022 \\
Research: Design and Fabricate first prototype (06/21 - 08/21)\\
Plan of Study: End Coursework for Master’s (10 courses) and complete Master’s Defense (05/22)\\\\

01/2023 - 12/2023 \\
Research: Validate prototype and complete first extramural experience (06/23 - 08/23);\\
Plan of Study: End Coursework for PhD (9 courses) \\\\

01/2024 - 12/2024 \\
Research: Optimize the prototype and complete a second site visit (06/24 - 08/24)\\
Plan of Study: Qualification Exams (05/2024) \\\\

01/2025 - 12/2025 \\
Research: Completion of ASGC project (08/2025)\\
Plan of Study: Ph.D. Defense (12/2025)


\clearpage

\section{\emph{Goals}} 

\small{\emph{Please describe your short and long term career goals (500 words or less)}}
\\
\\
\large
Through a career in engineering research, it is my hope to harness the capacity of science and technology to help others lead healthier and more fulfilling lives. 

My short term goal is to become a qualified, well-prepared engineering researcher with the knowledge, skill set and proven ability to solve challenging engineering problems. To this end, within the next two years, I will complete my Master’s thesis, during which time I will design and fabricate the wearable neuromuscular sensing device described in my research proposal. Concurrently, in the short term, I hope to serve the local Auburn community and larger national educational community by developing robotics-oriented educational hardware and software and by publishing a free online library of robotics-themed educational content for K-12 students and teachers. I intend to complete coursework in Biomechatronics, Machine Intelligence, and Neuromuscular Medicine which will best prepare me to succeed in my short term goal.

In the long term, I hope to dedicate my time and engineering background towards the common goal of expanding and establishing a long term human presence in orbit around Earth, as well as on the Lunar and Martian surface. Accomplishing this goal will require the collaborative effort of a great many engineers, scientists, researchers, and leaders, and I hope to make a contribution in the research area of multi-modal and proximate human-robot interaction. The need for intuitive and effective assistive robotic devices in space environments is evident in the numerous avoidable accidents and injuries reported by astronauts, and in the successful deployment and development of robotic devices for preventing accidents and assisting humans on Earth. By helping to develop of such devices for use in space environments, I will best be able to pursue my long term goal.

By working in the short term to achieve a high level of engineering knowledge and skill, and in the long term towards developing assistive robotic devices for human space exploration, I will be able to develop technologies and scientific principles that will drastically improve the lives of individuals with chronic neuromuscular injury like stroke and spinal cord injury. This course will therefore allow me to achieve my goal of helping others lead healthier and more fulfilling lives.

\clearpage

\section{\emph{NASA Alignment}} 

\small{\emph{Please describe how your work is aligned with NASA's Mission Directorates
(500 words or less)\\\\
NASA's Mission Directorates\\
Aeronautics Research Mission Directorate (https://www.nasa.gov/aeroresearch)\\
Human Exploration & Operations Mission Directorate (https://www.nasa.gov/directorates/heo/index.html)\\
Science Mission Directorate (https://science.nasa.gov)\\
Space Technology Mission Directorate (https://www.nasa.gov/directorates/spacetech/home/index.html)\\
}}\\\\
\large
My research interests are closely aligned with NASA’s Space Technology Mission Directorate (STMD), which pursues the development of “prototype systems, demonstrating key capabilities, and validating operational concepts for future human missions beyond low-Earth orbit… uniquely related to crew safety and mission operations in deep space”. 

The research I have proposed, a novel wearable device for neuromuscular sensing, is a prototype system designed to establish highly intuitive human-robot cooperation based on astronaut bio-signals, and is intended to enhance the safety of extravehicular activities by reducing effort and fatigue. This project is motivated by the injuries and accidents astronauts face due to EVA task difficulty and space suit stiffness, which must increase in order to establish an effective long term human presence in Lunar and Martian environments. NASA’s specific technology research goals include technologies to assist humans to safely and efficiently control robotic and autonomous assets and reduce overall demands on astronauts will therefore be advanced through this research.

The proposed device is not only in support of NASA’s immediate and long term goals, but holds merit in the advancement of human knowledge and understanding as well. The proposed device will make it possible to consider and understand the mechanisms by which the motor cortex adjusts to the influence of non-standard and microgravity environments. This knowledge will inform and enhance the development of life-support and human habitation systems for low-Earth orbit and for long term human habitation in space environments. The development of 3D printer sensors, another core facet of the proposed research will serve as a revolution in space and Earth-based operations in much the same way as 3D printing of hardware. 

The transformative potential of the proposed research will enable better and more robust robotic assistance not only in human space exploration, but also in a large variety of practical terrestrial endeavours. The need for more intuitive and effective computer-/robot- human interfacing is apparent in many industries and technology enabling this is the focus of much research investment. The device I propose would crosscut large portions of the private and public sectors including the medical field, where neuromuscular medicine and surgical training can be substantially augmented, industrial processes, where worker safety and manufacturing automation can be enhanced, vehicle driver safety, cognitive and psychological therapy, and many others. Thus the alignment of the primary contribution of this research with STMD’s interest in addressing the technology needs of NASA and the nation.



\end{document}